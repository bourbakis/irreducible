\documentclass{beamer}
\usepackage{amssymb,amsmath}
\mode<presentation>
{
  \usetheme{default}
  \usecolortheme{spruce}
  \usefonttheme{default} 
  \setbeamertemplate{navigation symbols}{}
  \setbeamertemplate{caption}[numbered]
} 

\title{Irreducible Polynomials: Examples}
\author{You}
\date{Date of Presentation}

\begin{document}

\begin{frame}
  \titlepage
\end{frame}

\begin{frame}{Introduction}
Here we give examples of irreducible(\textit{quadratic} or \textit{cubic}) in different rings.
\end{frame}


\begin{frame}{Irreducible quadratics in $\mathbb{Q}[x]$}
$f(x) = x^2+5$ is an irreducible quadratics in ring $\mathbb{Q}[x]$.

Since roots of $f$ are outside of real fields(in complex fields), we cannot factorize $f$ into products of linear polynomial(s).
\end{frame}


\begin{frame}{Irreducible quadratics in $\mathbb{Z}_p[x]$}
We take $f(x) = x^2+x+1$ and prime $p=2$.

Since $f([0])=[1]$ and $f([1])= [1]$.
$f$ is not divisible by a linear polynomial in $\mathbb{Z}_2$.

So the polynomial is irreducible.
\end{frame}


\begin{frame}{Irreducible cubics in $\mathbb{Z}_p[x]$}
Take $f(x)= x^3+2$ and prime $p=7$.
We can see that if $f$ is not irreducible, then at least one (\textit{up to a scalar}) linear polynomial divides $f$. So $f$ has a root in field $\mathbb{Z}_7$.

But in fact $f(0)=2$,$f(1)=f(2)=f(4)=3$, and $ff(3)=f(5)=f(6)=1$, $f(x)=0$ is insolvable in field $\mathbb{Z}_7$.

We kick out all possible ways, so $f$ must be irreducible.

\end{frame}



\begin{frame}{Irreducible cubics in $\mathbb{Z}_p[x]$}
Actually, $x^3+2$ is reducbile in $\mathbb{Z}_3$.

In fact,

$$
x^3+2 = x^3-1 = (x-1)(x^2+x+1)
$$
\end{frame}


\begin{frame}{Appendix: basic facts about number fields.}

\begin{itemize}
    \item $\mathbb{Q}$ is the field of rational numbers.
    
    \item $\mathbb{Z}_p$ is one of finite fields, \textit{i.e.} consisting only finite number of elements:
    
    $$
    \Bar{0},\Bar{1},\dots,\Bar{p-1}.
    $$
\end{itemize}

\end{frame}


\end{document}
